\chapter{Introdu\c{c}\~ao}

Perl \'e uma linguagem de programa\c{c}\~ao de alto n\'ivel, usada em aplica\c{c}\~oes Web e Desktop. Ela foi desenvolvida por Larry Wall
em 1987, a sigla PERL significa ''\textit{Practical Extraction And Report Language}'' que traduzindo equivale a ''Linguagem Pr\'atica de Relat\'orio e 
Extra\c{c}\~ao''. Perl destaca-se por ser r\'apida, eficiente e de f\'acil manuten\c{c}\~ao. 

A comunidade Perl reuniu m\'odulos, classes, scripts e frameworks no CPAN (\textit{Comprehensive Perl Archive   Network}), reposit\'orio onde voc\^e pode 
encontrar quase tudo j\'a desenvolvido na linguagem. Perl também tornou-se muito popular fora do Brasil por ser uma linguagem que previne erros de 
seguran\c{c}a, sendo portanto muito pouco prov\'avel que voc\^e cometa algum erro de implementa\c{c}\~ao que comprometa a sua aplica\c{c}\~ao. 


